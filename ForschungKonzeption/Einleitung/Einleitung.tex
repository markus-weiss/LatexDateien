\section{Zur Einleitung}
Im Folgenden ist der Weg zur Einleitung, wie auch die resultierende Einleitung aufgeführt.
\chapter{Checkliste}


\begin{itemize}
\item Hinführung
\item Ausgangssituation
\begin{itemize}
	\item Kurz das Thema schildern.
	\item Hierraus wird der Sollzustand abgeleitet werden können.
\end{itemize}
\item Problem 
\begin{itemize}
	\item Diskrepanz bzw. die Abweichung zwischen \textit{Soll} und \textit{Ist} 						verdeutlichen.
	\item Ist Situation formulieren. 
	\item Dient später zur Formulierung des Ziels.
\end{itemize}
\item Zielformulierung ( \textit{In einem Satz, wenn möglich keine Frage.} )
\begin{itemize}
	\item Abgleichen mit der Zielkategorie. Also: Beschreiben, Erkennen oder Gestalten.
	\item \textit{Soll}- Situation formulieren.
	\item Soll das oben genannte Problem lösen können. 
\end{itemize}
\item Vorhergehensweise
\begin{itemize}
	\item Beschreibung der Gliederung(Da die Gliederung ja den Roten Faden wiedergibt).
	\item Orientierung an der Gliederung.
	\item Offene Schrittfolgen Begründen.
\end{itemize}
\end{itemize}



\chapter{Einleitung}

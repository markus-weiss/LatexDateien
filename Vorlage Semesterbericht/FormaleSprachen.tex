\chapter{Formale Sprachen}\label{sec:Kapitel formale Sprachen}
\section{Grundlagen der Mengenlehre}\label{sec:Einführung_FS}

\textbf{Kapitelinhalt:}
\begin{itemize}
\item Grundlagen der Mengenlehre
\item Relationen und Funktionen
\item Die Welt der Zahlen
\item Rekursion und induktive Beweise 
\end{itemize}


\subsection{Der Mengenbegriff}


\begin{figure}[h]
\centering
\textit{a} $\in$ M,\\ bedeutet a ist \textbf{ein} Element der Menge \textit{M}.\\
\textit{a} $\notin$ M,\\ bedeutet a ist \textbf{kein} Element der Menge \textit{M}.\\
\textit{a,b} $\in$ M,\\ bedeutet, dass a und b \textbf{ein} Element der Menge \textit{M} sind.\\
\textit{a,b} $\notin$ M,\\ bedeutet, dass a und b \textbf{keine} Element der Menge \textit{M} sind.\\

\textit{M}\textsubscript{1} und \textit{M}\textsubscript{2} gelten als \textbf{gleich}\\ 
(\textit{M}\textsubscript{1} $=$ \textit{M}\textsubscript{2}),\\
 wenn sie \textbf{exakt} dieselben Elemente enthalten.\\

\textit{M}\textsubscript{1} und \textit{M}\textsubscript{2} gelten als \textbf{ungleich}\\ 
(\textit{M}\textsubscript{1} $\neq$ \textit{M}\textsubscript{2}),\\
 wenn sie \textbf{exakt} dieselben Elemente enthalten.\\

Auch eine \textit{leere} Menge, gilt als Menge und wird mit $\emptyset$ symbolisiert.\\
 
In einer Menge ist \textbf{niemals zweimal} das selbe Element enthalten und besitzen auch keinen festen Platz. Sie sind somit \textit{inhärent} und \textit{ungeordnet}.\\

\end{figure}


\newpage    

\textbf{Zeichenerklärung:}\\

\textbf{Runde Klammern:}\\

\begin{figure}[h]
\centering
{(...)} = Tuple o. Paar.\\ 
\end{figure}

(\textit{Hinweis:} In Runden Klammern ist die Reihenfolge entscheidet. D.h. das dieses Paar auch nur genau so als Menge vorkommen darf )\\

\textbf{Geschweifte Klammern:} \\ 

\begin{figure}[h]
\centering
\{...\} = Aufzählung. \\ 
\end{figure}

(\textit{Hinweis:} Bei einer Aufzählung von Mengen ist die Reihenfolge egal.\\




\textbf{Aufzählende Beschreibung}\\
Die Elemente einer Menge werden explizit aufgelistet. Selbst unendliche
Mengen lassen sich aufzählend \textit{enumerativ} beschreiben,
wenn die Elemente einer unmittelbar einsichtigen Regelmäßigkeit
unterliegen. Die nachstehenden Beispiele bringen Klarheit:\\

Die Menge der \textit{natürlichen Zahlen}: \\

\begin{figure}[h]
\centering

$\mathbb{N}$ := \{0,1,2,3,... \}\\
\textit{M}\textsubscript{1} := \{0,1,2,3,... \}\\
\end{figure}

\textbf{Deskriptive Beschreibung}\\
Die Mengenzugehörigkeit eines Elements wird durch eine charakteristische
Eigenschaft beschrieben. Genau jene Elemente sind in der
Menge enthalten, auf die die Eigenschaft zutrifft.\\

\begin{figure}[h]
\centering
\textit{M}\textsubscript{3} := \{ \textit{n} $\neq$ $\mathbb{N}$ | \textit{n} mod 2 = 0 \}\\
\textit{M}\textsubscript{4} := \{ \textit{n} \textsuperscript{2} | $\neq$ $\mathbb{N}$  \}\\
\end{figure}

Demnach enthält die Menge \textit{M}\textsubscript{3} alle Elemente \textit{n} $\in$ $\mathbb{N}$, die sich ohne
Rest durch 2 dividieren lassen, und die Menge \textit{M}\textsubscript{4} die Werte \textit{n} \textsuperscript{2} für alle natürlichen Zahlen \textit{n} $\in$ $\mathbb{N}$. Die Mengen \textit{M}\textsubscript{3} und \textit{M}\textsubscript{4} sind damit nichts anderes als eine deskriptive Beschreibung der im vorherigen
Beispiel eingeführten Mengen \textit{M}\textsubscript{1} und \textit{M}\textsubscript{2}.\\

\textbf{Teilmengenbeziehungen:}
\begin{figure}[H]
\centering

$\subseteq$ = ist Teilmenge von:\\ 
\textit{M}\textsubscript{1} $\subseteq$ \textit{M}\textsubscript{2} 
$\Leftrightarrow$ 
Aus \textit{a} $\in$ \textit{M}\textsubscript{1} flogt \textit{a} $\in$ \textit{M}\textsubscript{2}\\


$\supseteq$ = ist Obermenge von:\\
\textit{M}\textsubscript{1} $\subseteq$ \textit{M}\textsubscript{2}
$\Leftrightarrow$ 
\textit{M}\textsubscript{2} $\supseteq$ \textit{M}\textsubscript{1}\\

\end{figure}


\subsection{Mengenoperationen}

Vereinigung
Schnitt
Differenz
Komplement

Potenzmenge

Kardinalität

Kommunikativgesetze
Distributivgesetze
Neutrales Element
Inverse Elemente
Assoziativgesetze
Idempotenzgesetze
Absorptionsgesetze
Gesetze von De Morgan
Auslöschungsgesetze
Gesetz der Doppelnegation

\section{Relationen und Funktionen}

\textbf{Zeichenerklärung:}\\
\begin{figure}[h]
\centering
\textbf{Relationalzeichen:} \textit{$\sim$} = steht in Relation zu.\\
\end{figure}

Kartesischses Produkt

\subsection{Relation}

Graphdarstellung
Matrix-Darstellung hier noch bespiel der Potentmenge als Matrix (alle möglichen kombinationen müssen abgedeckt sein)\\


Relationsattribute:
reflexiv
irreflexiv
symetisch
asymmetisch
antisymmetisch
transitiv
linkstotal
rechttotal
linkseindeutig
rechtseindeutig

Relationsprodukt, inverse Relation

Transitive Hülle

Reflexiv Transitive Hülle

Äquivalenzrelation
Ordnungsrelation


Surjetivität
Injetivität
Bijetivität


\subsection{Funktion}

Bei Funktionen kann zunächst zwischen Funktionen der Informatik und der Mathematik gesprochen werden. 

total 
partiell

surjetiv
bijetiv
bijetiv


\section{Die Welt der Zahlen}
\subsection{Natürliche, rationale und reelle Zahlen}
\subsection{Von großen Zahlen}
\subsection{Die Unendlichkeit begreifen}

\section{Rekursion und induktive Beweise}
\subsection{Vollständige Induktion}






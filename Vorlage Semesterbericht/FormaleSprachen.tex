\chapter{Formale Sprachen}\label{sec:Kapitel formale Sprachen}
\section{Grundlagen der Mengenlehre}\label{sec:Einführung_FS}

\textbf{Kapitelinhalt:}
\begin{itemize}
\item Grundlagen der Mengenlehre
\item Relationen und Funktionen
\item Die Welt der Zahlen
\item Rekursion und induktive Beweise 
\end{itemize}


\subsection{Der Mengenbegriff}


\begin{figure}[h]
\centering
\textit{a} $\in$ M,\\ bedeutet a ist \textbf{ein} Element der Menge \textit{M}.\\
\textit{a} $\notin$ M,\\ bedeutet a ist \textbf{kein} Element der Menge \textit{M}.\\
\textit{a,b} $\in$ M,\\ bedeutet, dass a und b \textbf{ein} Element der Menge \textit{M} sind.\\
\textit{a,b} $\notin$ M,\\ bedeutet, dass a und b \textbf{keine} Element der Menge \textit{M} sind.\\

\textit{M}\textsubscript{1} und \textit{M}\textsubscript{2} gelten als \textbf{gleich}\\ 
(\textit{M}\textsubscript{1} $=$ \textit{M}\textsubscript{2}),\\
 wenn sie \textbf{exakt} dieselben Elemente enthalten.\\

\textit{M}\textsubscript{1} und \textit{M}\textsubscript{2} gelten als \textbf{ungleich}\\ 
(\textit{M}\textsubscript{1} $\neq$ \textit{M}\textsubscript{2}),\\
 wenn sie \textbf{exakt} dieselben Elemente enthalten.\\

Auch eine \textit{leere} Menge, gilt als Menge und wird mit $\emptyset$ symbolisiert.\\
 
In einer Menge ist \textbf{niemals zweimal} das selbe Element enthalten und besitzen auch keinen festen Platz. Sie sind somit \textit{inhärent} und \textit{ungeordnet}.\\

\end{figure}


\newpage    

\textbf{Zeichenerklärung:}\\

\textbf{Runde Klammern:}\\

\begin{figure}[h]
\centering
{(...)} = Tuple o. Paar.\\ 
\end{figure}

(\textit{Hinweis:} In Runden Klammern ist die Reihenfolge entscheidet. D.h. das dieses Paar auch nur genau so als Menge vorkommen darf )\\

\textbf{Geschweifte Klammern:} \\ 

\begin{figure}[h]
\centering
\{...\} = Aufzählung. \\ 
\end{figure}

(\textit{Hinweis:} Bei einer Aufzählung von Mengen ist die Reihenfolge egal.\\




\textbf{Aufzählende Beschreibung}\\
Die Elemente einer Menge werden explizit aufgelistet. Selbst unendliche
Mengen lassen sich aufzählend \textit{enumerativ} beschreiben,
wenn die Elemente einer unmittelbar einsichtigen Regelmäßigkeit
unterliegen. Die nachstehenden Beispiele bringen Klarheit:\\

Die Menge der \textit{natürlichen Zahlen}: \\

\begin{figure}[h]
\centering

$\mathbb{N}$ := \{0,1,2,3,... \}\\
\textit{M}\textsubscript{1} := \{0,1,2,3,... \}\\
\end{figure}

\textbf{Deskriptive Beschreibung}\\
Die Mengenzugehörigkeit eines Elements wird durch eine charakteristische
Eigenschaft beschrieben. Genau jene Elemente sind in der
Menge enthalten, auf die die Eigenschaft zutrifft.\\

\begin{figure}[h]
\centering
\textit{M}\textsubscript{3} := \{ \textit{n} $\neq$ $\mathbb{N}$ | \textit{n} mod 2 = 0 \}\\
\textit{M}\textsubscript{4} := \{ \textit{n} \textsuperscript{2} | $\neq$ $\mathbb{N}$  \}\\
\end{figure}

Demnach enthält die Menge \textit{M}\textsubscript{3} alle Elemente \textit{n} $\in$ $\mathbb{N}$, die sich ohne
Rest durch 2 dividieren lassen, und die Menge \textit{M}\textsubscript{4} die Werte \textit{n} \textsuperscript{2} für alle natürlichen Zahlen \textit{n} $\in$ $\mathbb{N}$. Die Mengen \textit{M}\textsubscript{3} und \textit{M}\textsubscript{4} sind damit nichts anderes als eine deskriptive Beschreibung der im vorherigen
Beispiel eingeführten Mengen \textit{M}\textsubscript{1} und \textit{M}\textsubscript{2}.\\

\textbf{Teilmengenbeziehungen:}
\begin{figure}[H]
\centering

$\subseteq$ = ist Teilmenge von:\\ 
\textit{M}\textsubscript{1} $\subseteq$ \textit{M}\textsubscript{2} 
$\Leftrightarrow$ 
Aus \textit{a} $\in$ \textit{M}\textsubscript{1} flogt \textit{a} $\in$ \textit{M}\textsubscript{2}\\


$\supseteq$ = ist Obermenge von:\\
\textit{M}\textsubscript{1} $\subseteq$ \textit{M}\textsubscript{2}
$\Leftrightarrow$ 
\textit{M}\textsubscript{2} $\supseteq$ \textit{M}\textsubscript{1}\\

\end{figure}


\subsection{Mengenoperationen}


\textbf{Vereinigung:}\textit{M}\textsubscript{1} vereinigt mit \textit{M}\textsubscript{2}\\
\textit{M}\textsubscript{1} $\cup$ \textit{M}\textsubscript{2} :=  \\
\{ \textit{a} | \textit{a} $\in$ \textit{M}\textsubscript{1} oder \textit{a} $\in$ \textit{M}\textsubscript{2} \}


\textbf{Schnitt:} \textit{M}\textsubscript{1} (herraus)geschnitten \textit{M}\textsubscript{2} 
\textit{M}\textsubscript{1} $\cap$ \textit{M}\textsubscript{2} :=  \\
\{ \textit{a} | \textit{a} $\in$ \textit{M}\textsubscript{1} und \textit{a} $\in$ \textit{M}\textsubscript{2} \}


\textbf{Differenz:} Die Differenz meint die Menge, den Anteil von \textit{M}\textsubscript{2} aus \textit{M}\textsubscript{1} entfernt.
\textit{M}\textsubscript{1} $\cap$ \textit{M}\textsubscript{2} :=  \\
\{ \textit{a} | \textit{a} $\in$ \textit{M}\textsubscript{1} und \textit{a} $\in$ \textit{M}\textsubscript{2} \}


\textbf{Komplement:} Die Komplementmenge, meint die Menge, die übrig bleibt wenn man alles weg lässt außer \textit{M}\textsubscript{1}.\\



\newpage
\subsubsection{Mengenalgebra}

\textbf{Kommunikativgesetze:} Meint, es ist egal wo eine Menge steht, die Operation ist die selbe.\\
\begin{figure}[H]
\centering
\textit{M}\textsubscript{1} $\cap$ \textit{M}\textsubscript{2} = 
\textit{M}\textsubscript{2} $\cap$ \textit{M}\textsubscript{1}\\

\textit{M}\textsubscript{1} $\cup$ \textit{M}\textsubscript{2} = 
\textit{M}\textsubscript{2} $\cup$ \textit{M}\textsubscript{1} \\
\end{figure}


\textbf{Distributivgesetze:} Meint, das eine Menge immer in eine Klammer hinein multipliziert werden kann. Wobei hier die Operatoren in der Klammer umgedreht werden müssen.\\
\begin{figure}[H]
\centering
\textit{M}\textsubscript{1} $\cup$ ( \textit{M}\textsubscript{2} $\cup$ \textit{M}\textsubscript{3}) =
(\textit{M}\textsubscript{1} $\cup$ \textit{M}\textsubscript{2}) $\cap$ (\textit{M}\textsubscript{1} $\cup$ \textit{M}\textsubscript{3})
\end{figure}


\textbf{Neutrales Element:} Das neutrale Element meint, eine Menge vereinigt mit der leeren Menge, ist die Menge selbst. Eine Menge geschnitten mit dem Komplement, ist auch die Menge selbst. \\
\begin{figure}[H]
\centering
\textit{M} $\cup$ $\emptyset$ = \textit{M}\\
\textit{M} $\cap$ \textit{T} = \textit{M}
\end{figure}




\textbf{Inverse Elemente:} Das inverse Element einer Menge ist, eine Menge vereinigt mit seiner \textit{Komplementärmenge} ist das \textit{Komplement}. Eine Menge geschnitten mit seiner Komplementärmenge, ist die \textit{leere Menge}.\\
\begin{figure}[H]
\centering
%\textit{M} $\cup$ \overline{\textit{M}} = \textit{T}\\
%\textit{M} $\cap$ \overline{\textit{M}} = $\emptyset$
\end{figure}

\textbf{Assoziativgesetze:} Meint, dass wenn auf alle Mengen einer Gleichung die selbe Operation durchgeführt wird, tritt wieder das Kommunikativgesetz in Kraft. Gilt also auch für Vereinigung wie auch für Schnitt\\
\begin{figure}[H]
\centering
\textit{M}\textsubscript{1} $\cup$ (\textit{M}\textsubscript{2} $\cup$ \textit{M}\textsubscript{3}) = 
(\textit{M}\textsubscript{1}) $\cup$ \textit{M}\textsubscript{2}) $\cup$ \textit{M}\textsubscript{3}\\
\end{figure}

\textbf{Idempotenzgesetze:} Eine Menge, geschnitten oder vereinigt mit sich selbst ist immer die Menge selbst.\\
\begin{figure}[H]
\centering
\textit{M} $\cup$ \textit{M} = \textit{M}\\
\textit{M} $\cap$ \textit{M} = \textit{M}\\
\end{figure}

\textbf{Absorptionsgesetze:} Eine Menge \textit{vereinigt} mit einem \textit{Schnitt} derselben Menge und einer anderen Menge ergibt die Menge selbst. Eine Menge \textit{geschnitten} mit einer \textit{Vereinigung} derselben Menge und einer anderen Menge ist auch die Menge selbst. \\
\begin{figure}[H]
\centering
\textit{M}\textsubscript{1} $\cup$ (\textit{M}\textsubscript{1} $\cap$ \textit{M}\textsubscript{2}) = \textit{M}\textsubscript{1}\\ 
\textit{M}\textsubscript{1} $\cap$ (\textit{M}\textsubscript{1} $\cup$ \textit{M}\textsubscript{2}) = \textit{M}\textsubscript{1}\\ 
\end{figure}

\textbf{Gesetze von De Morgan:} Das \textit{Komplement} einer Operation ist immer die gegenteilige Operation\\
%\begin{figure}[H]
%\centering
%\overline{\textit{M}\textsubscript{1} $\bar{\cup}$ \overline{\textit{M}\textsubscript{2}} = 
%\overline{\textit{M}\textsubscript{1}} $\cap$ \overline{\textit{M}\textsubscript{2}}
%\end{figure}

\textbf{Auslöschungsgesetze:} \\
Eine Menge vereinigt mit der \textit{Komplementärmenge} ist die Komplementärmenge.\\
\begin{figure}[H]
\centering
\textit{M} $\cup$ \textit{T} = \textit{T}\\
\end{figure}
Eine Menge geschnitten mit der neutralen Menge ist die neutrale Menge.\\
\begin{figure}[H]
\centering
\textit{M} $\cup$ $\emptyset$ = $\emptyset$\\
\end{figure}

\textbf{Gesetz der Doppelnegation:}\\
\begin{figure}[H]
\centering
$\overline{\overline{M}}$ = \textit{M}
\end{figure}


\textbf{Kardinalität:} Meint den Betrag deiner Menge. Also die Anzahl der Elemente einer Menge.\\
\begin{figure}[H]
\centering
\textit{M} = \{1,2,3,4\}  $\Rightarrow$ |\textit{A}| =4\\
\end{figure}

\textbf{Potenzmenge:} Meint die \textit{Vereinigung} aller Teilmengen zu einer neuen Menge 2\textsuperscript{\textit{M}}. Jede nicht leere Menge hat mindestens Zwei Elemente\\

\textit{M} = \{ \{a\}  ,\{b\} \}
\textit{P} = \{ $\emptyset$, \{a\}  ,\{b\}, \{a,b\} \}\\
2\textsuperscript{ \textit{M}} := \{ \textit{M'} | \textit{M'} $\subset$ \textit{M} \}


\textbf{Partition \& Äquivalenzklassen:} Eine \textit{Partition} von \textit{M}, ist wenn jedes Element aus M in nur einer Menge aus \textit{P} liegt. Die Elemente aus \textit{P} werden als \textit{Äquivalenzklassen} bezeichnet.\\
\textit{P} $\subseteq$ 2\textsuperscript{\textit{M}}

\section{Relationen und Funktionen}

\textbf{Zeichenerklärung:}\\
\begin{figure}[h]
\centering
\textbf{Relationalzeichen:} \textit{$\sim$} = steht in Relation zu.\\
\end{figure}

\textbf{Relation:} (sin tuple, Beziehungen innerhalb einer Menge) Eine Relation meint, wenn zwei Elemente in einer \textit{Relation} oder auch \textit{Zusammenhang} stehen. So meint x \~ y das diese Elemente bzgl. der Relation in Zusammenhang stehen. Das Gegenteil x NICHT\~ y stehen nicht in Zusammenhang. Relation ist eine Teilmenge eines karesicehn Produkt einer Menge\\

\textbf{Kartesischses Produkt:} Das Kartesische produkt von M x M bilded man tupel in der Menge selbst. Man bildet alle Elemente auf alle ab. 

\subsection{Relation}

Graphdarstellung
Matrix-Darstellung hier noch bespiel der Potentmenge als Matrix (alle möglichen kombinationen müssen abgedeckt sein)\\


Relationsattribute:\\

reflexiv: wenn alle Elemente einer Menge M mit sich selbst verbunden sind.\\

irreflexiv: Wenn kein Element einer Menge M mit sich selbst verbunden ist.\\

symetisch: Wenn (für alle Elemente) x mit y und auch y mit x verbunden ist.\\

asymmetisch: Wenn x - y aber nicht y - x verbunden ist\\

antisymmetisch: Wenn für ein Element x - y gibt un dieses auch y - x folgt das x = y. Wenn gewisse Elemente tuple in einer Matrix mit sich selbst verbunden sind und auch gleich sie selbst sind.\\

transitiv: Wenn es einen Weg von 1 - 2 und 2 - 3 gibt, gibt es auch immer einen (indirekten) Weg von 1-3\\
 
http://www.mathematik.net/relationen/re1s0.htm
 
 
linkstotal (linksvollständig): Wenn für alle x genau eine Verbindung zu einem y besteht.\\

rechttotal(rechtsvollständig): Wenn für alle y genau eine Verbindung zu einem x besteht.\\

linkseindeutig: Wenn x \~ z und y \~ y folgt das x = y. Wenn x durch y ersetzt werden kann weil das einzige was sie definiert ihre verbindung zu z ist.\\


http://www.mathematik.net/relationen/re2s0.htm
rechtseindeutig: 
 Wenn x \~ y und x \~ z folgt das x = z. Es gibt keinen Pfeil der zweimal auf ein element von x nach y zeigt

Relationsprodukt, inverse Relation:  \\
R*S = (x,y) | es gibt ein z mit x \~ z und z \~y.
R^0 := (x,x)\\ 
R^n :=R*R^n-1 = 2^n = 2 * 2^n-1\\

R^0 = reflexiv, alle Elemente sind mit sich selbst verbunden.\\

 \textit{Gleichheitsrelation} oder \textit{Identität}

Transitive Hülle: Kleinste gemeinsame Verbindung\\

Reflexiv Transitive Hülle: Die geringste Menge an verbindungen und die Verbindung der Elemente mit sich selbst.\\

Äquivalenzrelation: reflexiv, symmetrisch und transitiv\\
Jeder mit sich selbst, jede Verbindung in beide richtungen, Wenn von a nach b und b zu c dann a zu c\\

Ordnungsrelation: reflexiv, transitiv und antisymmetrisch \\
Jeder mit sich selbst, jede Verbindung in beide richtungen, Wenn x durch y ersetzt werden kann \\



Surjetivität: Wenn von a zu b mindestens eine verbindung besteht.\\

Injetivität: Es gibt von b zu a genau eine Abbildung\\


Bijetivität: Injetiv + Surjektiv, wenn jedes element von a und b genau eine Verbindung bestitzen eine 1:1 zuordnung.


\subsection{Funktion}

Bei Funktionen kann zunächst zwischen Funktionen der Informatik und der Mathematik gesprochen werden. 

total 
partiell

surjetiv
bijetiv
bijetiv


\section{Die Welt der Zahlen}
\subsection{Natürliche, rationale und reelle Zahlen}
\subsection{Von großen Zahlen}
\subsection{Die Unendlichkeit begreifen}

\section{Rekursion und induktive Beweise}
\subsection{Vollständige Induktion}






%Dokumentklasse
\documentclass[a4paper,12pt]{scrreprt}
\usepackage[left= 2.5cm,right = 2cm, bottom = 4 cm]{geometry}
%\usepackage[onehalfspacing]{setspace}
% ============= Packages =============

% Dokumentinformationen
\usepackage[
	pdftitle={Semesterbericht},
	pdfsubject={Wissenschaftliches Arbeiten},
	pdfauthor={Markus Weiß},
	pdfkeywords={},	
	%Links nicht einrahmen
	hidelinks
]{hyperref}



% Standard Packages
\usepackage[utf8]{inputenc}
\usepackage{float} 
\usepackage[ngerman]{babel}
\usepackage[T1]{fontenc}
\usepackage{graphicx, subfig}
\usepackage{fancyhdr}
\usepackage{lmodern}
\usepackage{color}
\usepackage{amsmath}

%Graficpath
\graphicspath{{img/}}

% zusätzliche Schriftzeichen der American Mathematical Society
\usepackage{amsfonts}
\usepackage{amsmath}

%nicht einrücken nach Absatz
%\setlength{\parindent}{0pt}


% ============= Kopf- und Fußzeile =============
\pagestyle{fancy}
%
\lhead{}
\chead{}
\rhead{\slshape \leftmark}
%%
\lfoot{}
\cfoot{\thepage}
\rfoot{}
%%
\renewcommand{\headrulewidth}{0.4pt}
\renewcommand{\footrulewidth}{0pt}

% ============= Package Einstellungen & Sonstiges ============= 
%Besondere Trennungen
\hyphenation{De-zi-mal-tren-nung}


% ============= Dokumentbeginn =============

\begin{document}

%Einrücken verhindern 
\setlength{\parindent}{0em}

%Seiten ohne Kopf- und Fußzeile sowie Seitenzahl
\pagestyle{empty}

% Beendet eine Seite und erzwingt auf den nachfolgenden Seiten die Ausgabe aller Gleitobjekte (z.B. Abbildungen), die bislang definiert, aber noch nicht ausgegeben wurden. Dieser Befehl fügt, falls nötig, eine leere Seite ein, sodaß die nächste Seite nach den Gleitobjekten eine ungerade Seitennummer hat. 
\cleardoubleoddpage

% pagestyle für gesamtes Dokument aktivieren
\pagestyle{fancy}


%===========================TITEL-Seite==============================

\title{Semesterbericht}
\date{15.01.2019}
\author{
Markus Weiß\\
Fakultät Digitale Medien,\\
Hochschule Furtwangen University\\
}
\maketitle

%Inhaltsverzeichnis
\tableofcontents

\chapter{Wissenschaftliches Arbeiten}\label{sec:parm}

\section{Forschungskonzeption}\label{sec:parm}

{\textsc{
Forschungskonzeption: Gedanklicher Entwurf eines Forschungsvorhabens\\
}}

\section{Forschungvorhaben}\label{sec:parm}

{\textsc{
Forschungsvorhaben: Wissen Generieren das wissenschaftlich begründet ist\\
}}

\section{Forschungsprojekt}\label{sec:parm}

\begin{itemize}
\item Themenfindung

\end{itemize}


\begin{itemize}
\item Strukturierung
\begin{itemize}
\item Literaturrecherche
\item Zeitplan
\item Forschungskonzeption
\end{itemize}

\end{itemize}

\begin{itemize}
\item Methoden
\begin{itemize}
\item Empirische Methoden
\item Theorien
\item Gestaltung
\end{itemize}
\end{itemize}

\begin{itemize}
\item Administratives
\begin{itemize}
\item Literatur- Quellensammlung (immer direkt übertragen)
\item Notizen- Ideensammlung
\item Sammlung von Textentwürfen (schreiben so früh wie möglich)
\end{itemize}
\end{itemize}

\begin{itemize}
\item Verfassen
\begin{itemize}
\item Texte schreiben
\item Risikopuffer
\item Rechtschreibung
\end{itemize}
\end{itemize}

\begin{itemize}
\item Controlling
\begin{itemize}
\item Neue Perspektiven
\item Forschungsfrage bzw. mit dem Ziel abgleichen
\item Abgleichen mit der Gliederung
\item Abgleichen mit dem Zeitplan
\item Risikoanalyse und Feedback einholen
\end{itemize}
\end{itemize}







\chapter{Advanced Media Production}
\section{Einführung}\label{sec:Einführung_AMP}

S3D = Stereoskopische 3D \\

\section{S3D‐Aufnahme‐, Übertragungs‐ und Darstellungstechnologien}

\subsection{Aufnahmetechnologien}
\begin{itemize}

\item Stereo‐Rendering in 3D‐Programmen und Game‐Engines: Maya Renderer, Unity S3D-Renderer
\item S3D‐Realfilm‐Kamerakonzepte 2015:
\begin{itemize}
\item 21stCentury3D/Hyperstereo
\item GoPro Dual Hero
\item P+S/Freestyle‐Spiegelrig
\end{itemize}

\end{itemize}

\subsection{Betrachtungstechnologien}
\begin{description}
\item Polfilterverfahren: Linear \& Zirkular\\
\\
	Projektoren werfen ihr Licht durch entweder recht und links zirkulierende Polfilter bis hin zur reflektierenden, Silber beschichteten Leinwand. Das Bild wird nun zurück geworfen und trifft auf die Polfilterbrille wo nun das recht drehende Licht, des rechten Bildes durch den recht gerichteten Zirkularfilter läuft und somit auf das rechte Auge trifft. Auf der linken Seite ist der Vorgang identisch.
	 
\item RGB Wellenmultiplexverfahren (z.B. CAVE-Installation) \\
\\
Hier werden die Farbinformationen nach Wellenlängen getrennt um jeweils ein rechtes und ein linkes Halbbild zu erhalten.

\item 3DTV-Polfilterverfahren
Hier werden jeweils die geraden und ungeraden Zeilen in die entgegengesetzte Richtung zirkular ausgerichtet.

\item 3DTV-Shutterverfahren
Auf dem Display wird jeweils das rechte und im Anschluss das rechte Bild angezeigt. Mit einer über Infrarot synchronisierte Shutterbrille wird jeweils das eine oder andere Auge zeit synchron das Bild erhalten.

\item Anaglyphen-Verfahren
Die linke Seite der Brille ist cyan, die rechte magenta gefärbt ...?

%=============================%
\item autostereoskope Displays\\
\begin{description}
\item Prallax‐Barriere Display\\
...
\item Lentikular Display
...
\end{description}
Script Folie 107
\item VR-Headset
\end{description}

\subsection{Displayspeisung->Bildorganisation}
\begin{itemize}
\item Frame Packing \textit{progessiv, interlace} \\

\item Side-by-Side \textsubscript{(Half)*}

\item Top-and-Bottom \textsubscript{(squeezed)*}
\end{itemize}

\subsection{Unser Auge ist keine Kamera}
\begin{enumerate}
\item Pupille \& Iris\\
Diese öffnet sich bei schwachem, und schließt sich bei starkem Lichteinfall.

\item Muskuläre Steuerung\\
Unserer Augäpfel sind muskulär steuerbar (3 Freiheitsgrade,
s.a. Folie 28) und im Blickfeld frei beweglich, zusätzlich
unterstützt durch Kopfbewegungen (Folie 30).

\item Akkommodation\\
Die Linse in unserem Auge verändert ihre Brennweite durch muskuläre Dehnung/Streckung
Ziel: maximale Schärfeabbildung eines gewünschten Fixationspunkts auf die fovea centralis -> Akkommodation.
Fokussierungen -> 4m Abstand -> Ziliarkörper ist komplett entspannt. Linse ist in den Zonularfasern natürlich aufgespannt (dünn)
Fokussierungen < 2m Abstand führen zu deutlich „spürbaren“ Ziliarkontraktionen. Linse wird in den Zonularfasern entspannt (dick)

\item Rezeptive Felder\\
Die Rezeptoren auf der gekrümmten
Netzhaut sind zusammengeschaltet zu
„Rezeptiven Feldern“, welche mit zunehmenden
Abstand vom Sehzentrum
immer größer werden. Ein Nervenimpuls
wird nur bei Helligkeitsdifferenzen innerhalb
des Feldes ausgelöst.


\item Fovea zentralis\\
Zentraler Schärfebereich (fovea
centralis) und temporales / nasales
Gesichtsfeld werden im Gehirn
getrennt verarbeitet (s. Folie 34).

\end{enumerate}


\subsection{Natürliche Okulomotorik}

Unser natürliches Sehen ist geprägt von einem Konvergenz-zu-Akkomodations-Verhältnis (C/A)-Ratio von $\approx$ 1:1

\begin{itemize}

\item \textbf{Konvergenz:} Das Eindrehen der Augen
\item \textbf{Fixation:} Konvergieren zum Kreuzungspunkt
\item \textbf{Akkommodation:} Ist die Konvergenz im Gange versuchen \textit{gleichzeitig} Ziliarmuskeln den Brechungsindex der Augenlinsen Reflexhaft anzupassen
\item \textbf{Okulomotorische Stereopsis:} Zerebrales Feedback der muskulären Spannungen von Konvergenz \&	Akkommodation und Abgleich mit erlerntem, absolutem Entfernungswissen (Funktioniert aber nur bis ca. 2m Entfernung, Das Hirn kann sich Anhand der Muskelspannung den Abstand merken)

\end{itemize}








\chapter{Formale Sprachen}\label{sec:Kapitel formale Sprachen}
\section{Grundlagen der Mengenlehre}\label{sec:Einführung_FS}

\textbf{Kapitelinhalt:}
\begin{itemize}
\item Grundlagen der Mengenlehre
\item Relationen und Funktionen
\item Die Welt der Zahlen
\item Rekursion und induktive Beweise 
\end{itemize}


\subsection{Der Mengenbegriff}


\begin{figure}[h]
\centering
\textit{a} $\in$ M,\\ bedeutet a ist \textbf{ein} Element der Menge \textit{M}.\\
\textit{a} $\notin$ M,\\ bedeutet a ist \textbf{kein} Element der Menge \textit{M}.\\
\textit{a,b} $\in$ M,\\ bedeutet, dass a und b \textbf{ein} Element der Menge \textit{M} sind.\\
\textit{a,b} $\notin$ M,\\ bedeutet, dass a und b \textbf{keine} Element der Menge \textit{M} sind.\\

\textit{M}\textsubscript{1} und \textit{M}\textsubscript{2} gelten als \textbf{gleich}\\ 
(\textit{M}\textsubscript{1} $=$ \textit{M}\textsubscript{2}),\\
 wenn sie \textbf{exakt} dieselben Elemente enthalten.\\

\textit{M}\textsubscript{1} und \textit{M}\textsubscript{2} gelten als \textbf{ungleich}\\ 
(\textit{M}\textsubscript{1} $\neq$ \textit{M}\textsubscript{2}),\\
 wenn sie \textbf{exakt} dieselben Elemente enthalten.\\

Auch eine \textit{leere} Menge, gilt als Menge und wird mit $\emptyset$ symbolisiert.\\
 
In einer Menge ist \textbf{niemals zweimal} das selbe Element enthalten und besitzen auch keinen festen Platz. Sie sind somit \textit{inhärent} und \textit{ungeordnet}.\\

\end{figure}


\newpage    

\textbf{Zeichenerklärung:}\\

\textbf{Runde Klammern:}\\

\begin{figure}[h]
\centering
{(...)} = Tuple o. Paar.\\ 
\end{figure}

(\textit{Hinweis:} In Runden Klammern ist die Reihenfolge entscheidet. D.h. das dieses Paar auch nur genau so als Menge vorkommen darf )\\

\textbf{Geschweifte Klammern:} \\ 

\begin{figure}[h]
\centering
\{...\} = Aufzählung. \\ 
\end{figure}

(\textit{Hinweis:} Bei einer Aufzählung von Mengen ist die Reihenfolge egal.\\




\textbf{Aufzählende Beschreibung}\\
Die Elemente einer Menge werden explizit aufgelistet. Selbst unendliche
Mengen lassen sich aufzählend \textit{enumerativ} beschreiben,
wenn die Elemente einer unmittelbar einsichtigen Regelmäßigkeit
unterliegen. Die nachstehenden Beispiele bringen Klarheit:\\

Die Menge der \textit{natürlichen Zahlen}: \\

\begin{figure}[h]
\centering

$\mathbb{N}$ := \{0,1,2,3,... \}\\
\textit{M}\textsubscript{1} := \{0,1,2,3,... \}\\
\end{figure}

\textbf{Deskriptive Beschreibung}\\
Die Mengenzugehörigkeit eines Elements wird durch eine charakteristische
Eigenschaft beschrieben. Genau jene Elemente sind in der
Menge enthalten, auf die die Eigenschaft zutrifft.\\

\begin{figure}[h]
\centering
\textit{M}\textsubscript{3} := \{ \textit{n} $\neq$ $\mathbb{N}$ | \textit{n} mod 2 = 0 \}\\
\textit{M}\textsubscript{4} := \{ \textit{n} \textsuperscript{2} | $\neq$ $\mathbb{N}$  \}\\
\end{figure}

Demnach enthält die Menge \textit{M}\textsubscript{3} alle Elemente \textit{n} $\in$ $\mathbb{N}$, die sich ohne
Rest durch 2 dividieren lassen, und die Menge \textit{M}\textsubscript{4} die Werte \textit{n} \textsuperscript{2} für alle natürlichen Zahlen \textit{n} $\in$ $\mathbb{N}$. Die Mengen \textit{M}\textsubscript{3} und \textit{M}\textsubscript{4} sind damit nichts anderes als eine deskriptive Beschreibung der im vorherigen
Beispiel eingeführten Mengen \textit{M}\textsubscript{1} und \textit{M}\textsubscript{2}.\\

\textbf{Teilmengenbeziehungen:}
\begin{figure}[H]
\centering

$\subseteq$ = ist Teilmenge von:\\ 
\textit{M}\textsubscript{1} $\subseteq$ \textit{M}\textsubscript{2} 
$\Leftrightarrow$ 
Aus \textit{a} $\in$ \textit{M}\textsubscript{1} flogt \textit{a} $\in$ \textit{M}\textsubscript{2}\\


$\supseteq$ = ist Obermenge von:\\
\textit{M}\textsubscript{1} $\subseteq$ \textit{M}\textsubscript{2}
$\Leftrightarrow$ 
\textit{M}\textsubscript{2} $\supseteq$ \textit{M}\textsubscript{1}\\

\end{figure}


\subsection{Mengenoperationen}

Vereinigung
Schnitt
Differenz
Komplement

Potenzmenge

Kardinalität

Kommunikativgesetze
Distributivgesetze
Neutrales Element
Inverse Elemente
Assoziativgesetze
Idempotenzgesetze
Absorptionsgesetze
Gesetze von De Morgan
Auslöschungsgesetze
Gesetz der Doppelnegation

\section{Relationen und Funktionen}

\textbf{Zeichenerklärung:}\\
\begin{figure}[h]
\centering
\textbf{Relationalzeichen:} \textit{$\sim$} = steht in Relation zu.\\
\end{figure}

Kartesischses Produkt

\subsection{Relation}

Graphdarstellung
Matrix-Darstellung hier noch bespiel der Potentmenge als Matrix (alle möglichen kombinationen müssen abgedeckt sein)\\


Relationsattribute:
reflexiv
irreflexiv
symetisch
asymmetisch
antisymmetisch
transitiv
linkstotal
rechttotal
linkseindeutig
rechtseindeutig

Relationsprodukt, inverse Relation

Transitive Hülle

Reflexiv Transitive Hülle

Äquivalenzrelation
Ordnungsrelation


Surjetivität
Injetivität
Bijetivität


\subsection{Funktion}

Bei Funktionen kann zunächst zwischen Funktionen der Informatik und der Mathematik gesprochen werden. 

total 
partiell

surjetiv
bijetiv
bijetiv


\section{Die Welt der Zahlen}
\subsection{Natürliche, rationale und reelle Zahlen}
\subsection{Von großen Zahlen}
\subsection{Die Unendlichkeit begreifen}

\section{Rekursion und induktive Beweise}
\subsection{Vollständige Induktion}














%Verzeichnis aller Bilder
%\listoffigures

%Verzeichnis aller Tabellen
%\listoftables

\end{document}

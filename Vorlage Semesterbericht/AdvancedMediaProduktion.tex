\chapter{Advanced Media Production}
\section{Einführung}\label{sec:Einführung_AMP}

S3D = Stereoskopische 3D. S3D muss gerecht der visuellen Wahrnehmung produziert werden\\


\section{S3D‐Aufnahme‐, Übertragungs‐ und Darstellungstechnologien}

\subsection{Aufnahmetechnologien}
\begin{itemize}

\item Stereo‐Rendering in 3D‐Programmen und Game‐Engines: Maya Renderer, Unity S3D-Renderer
\item S3D‐Realfilm‐Kamerakonzepte 2015:
\begin{itemize}
\item 21stCentury3D/Hyperstereo
\item GoPro Dual Hero
\item P+S/Freestyle‐Spiegelrig
\end{itemize}

\end{itemize}

\subsection{Betrachtungstechnologien}
\begin{description}
\item Polfilterverfahren: Linear \& Zirkular\\
\\
	Projektoren werfen ihr Licht durch entweder recht und links zirkulierende Polfilter bis hin zur reflektierenden, Silber beschichteten Leinwand. Das Bild wird nun zurück geworfen und trifft auf die Polfilterbrille wo nun das recht drehende Licht, des rechten Bildes durch den recht gerichteten Zirkularfilter läuft und somit auf das rechte Auge trifft. Auf der linken Seite ist der Vorgang identisch.
	 
\item RGB Wellenmultiplexverfahren (z.B. CAVE-Installation) \\
\\
Hier werden die Farbinformationen nach Wellenlängen getrennt um jeweils ein rechtes und ein linkes Halbbild zu erhalten.

\item 3DTV-Polfilterverfahren
Hier werden jeweils die geraden und ungeraden Zeilen in die entgegengesetzte Richtung zirkular ausgerichtet.

\item 3DTV-Shutterverfahren
Auf dem Display wird jeweils das rechte und im Anschluss das rechte Bild angezeigt. Mit einer über Infrarot synchronisierte Shutterbrille wird jeweils das eine oder andere Auge zeit synchron das Bild erhalten.

\item Anaglyphen-Verfahren
Die linke Seite der Brille ist cyan, die rechte magenta gefärbt ...?

%=============================%
\item autostereoskope Displays\\
\begin{description}
\item Prallax‐Barriere Display\\
...
\item Lentikular Display
...
\end{description}
Script Folie 107
\item VR-Headset
\end{description}

\subsection{Displayspeisung->Bildorganisation}
\begin{itemize}
\item Frame Packing \textit{progessiv, interlace} \\

\item Side-by-Side \textsubscript{(Half)*}

\item Top-and-Bottom \textsubscript{(squeezed)*}
\end{itemize}

\subsection{Unser Auge ist keine Kamera}

\textbf{Gemeinsamkeiten von Kamera und Auge:}\\ 

\begin{table}[H]
\centering
\begin{tabular}{|r|l|}
Kamera & Auge \\
Blende & Iris \\
Sensor & Zapfen, Stäbchen \\
Fokusfunktion & Fokusfunktion \\
\end{tabular}
\end{table}


\textbf{ Was sie nicht gemeinsam haben:}\\
\begin{itemize}
\item Die Kamera hat:
\begin{itemize}
	\item eine Zoomfunktion.
	\item ist auf dem Sensor überall gleich Scharf.
\end{itemize}
\item Das Auge hat:
\begin{itemize}
	\item mehr Freiheitsgrade. Drei von den Muskeln + Kopfbewegung.
\end{itemize}
\end{itemize}



\textbf{Weitere Eigenheiten des Auges:} \\
\begin{itemize}
\item \textbf{Pupille \& Iris:}\\
Diese öffnet sich bei schwachem, und schließt sich bei starkem Lichteinfall.

\item \textbf{Zapfen:} \textit{Hellsehen, Scharfsehen} 

\begin{itemize}
	\item Zapfen sitzen ausschließlich in der Fovea Centralis
	\item Arbeiten ab ca. 1 clr/m\textsuperscript{2}	
	\item Es existieren drei Arten von Zapfen:
\begin{enumerate}
	\item SW: Short Wave
	\item MW: Medium Wave
	\item LW: Long Wave
\end{enumerate}
\end{itemize}

\item \textbf{Stäbchen:}  \textit{Dunkelsehen}
\begin{itemize}
	\item Arbeiten ab ca. <1 clr/m\textsuperscript{2}
	\item Stäbchen befinden sich überall anders, sind aber nach außen hin zu rezeptiven Feldern zusammengefasst. \textit{Siehe weiter unten Rezeptive Felder: }
	\item Zapfen 
\end{itemize}
\end{itemize}


\begin{itemize}

\item \textbf{Muskuläre Steuerung:}\\
Unserer Augäpfel sind muskulär steuerbar (3 Freiheitsgrade,
s.a. Folie 28) und im Blickfeld frei beweglich, zusätzlich
unterstützt durch Kopfbewegungen (Folie 30).

\item \textbf{Akkommodation:} \\
Die Linse in unserem Auge verändert ihre Brennweite durch muskuläre Dehnung/Streckung Ziel: maximale Schärfeabbildung eines gewünschten Fixationspunkts auf die fovea centralis -> Akkommodation.
Fokussierungen -> 4m Abstand -> Ziliarkörper ist komplett entspannt. Linse ist in den Zonularfasern natürlich aufgespannt (dünn) Fokussierungen < 2m Abstand führen zu deutlich „spürbaren“ Ziliarkontraktionen. Linse wird in den Zonularfasern entspannt (dick)

\item \textbf{Rezeptive Felder:} \textit{(Bestehend aus Stäbchen)}\\ 
 Die Rezeptoren auf der gekrümmten Netzhaut sind zusammengeschaltet zu „Rezeptiven Feldern“, welche mit zunehmenden Abstand vom Sehzentrum immer größer werden. Ein Nervenimpuls wird nur bei Helligkeitsdifferenzen innerhalb des Feldes ausgelöst des weiteren werden die rezeptiven Felder zur Kantenerkennung genutzt(Hell, Dunkel Kontraste).

\item \textbf{Fovea zentralis:}\\
Zentraler Schärfebereich (fovea centralis) und temporales / nasales Gesichtsfeld werden im Gehirn getrennt verarbeitet (s. Folie 34). Erkennung von Kanten und Bewegung Fluchtreflex
\end{itemize}


\subsection{Natürliche Okulomotorik}

Unser natürliches Sehen ist geprägt von einem Konvergenz-zu-Akkomodations-Verhältnis (C/A)-Ratio von $\approx$ 1:1

\begin{itemize}

\item \textbf{Konvergenz:} Das Eindrehen der Augen
\item \textbf{Fixation:} Konvergieren zum Kreuzungspunkt
\item \textbf{Akkommodation:} Ist die Konvergenz im Gange versuchen \textit{gleichzeitig} Ziliarmuskeln den Brechungsindex der Augenlinsen Reflexhaft anzupassen
\item \textbf{Okulomotorische Stereopsis:} Zerebrales Feedback der muskulären Spannungen von Konvergenz \&	Akkommodation und Abgleich mit erlerntem, absolutem Entfernungswissen aus Ringmuskel und Augenmuskeln. (Funktioniert aber nur bis ca. 2m Entfernung, Das Hirn kann sich Anhand der Muskelspannung den Abstand merken)

\item \textbf{Da wo wir hin konvergieren da Akkomodieren wir auch. im Verhältnis von 1:1}

\end{itemize}


\section{Warum sehen wir die Welt nicht doppelt}


\subsection{Wie nehmen wir unsere Umwelt war}
\textbf{ Unser Blickfeld besteht aus drei unter Kategorien:}

\begin{itemize}
\item  \textbf{Blickfeld 1:}
Bereich höchster örtlicher Auflösung -> optimiert auf Detailwahrnehmung
\item  \textbf{Blickfeld 2:}
Bereich differenzieller Grobverarbeitung -> optimiert auf Strukturwahrnehmung
\item  \textbf{Blickfeld 3:} 
Bereich flächiger, temporaler Verarbeitung -> optimiert auf Bewegungswahrnehmung
\end{itemize}

\textbf{Merksatz:} \\

Nur eine schmaler Bereich unserer visuellen Wahrnehmung realisiert hohe örtliche Auflösung.
Die übrigen Bereiche sind evolutionär optimiert auf die Erkennung von Grobstrukturen und Bewegungen!
Erst durch die Summe der muskulären Augensprünge („Saccaden“) mit jeweiliger Akkommodation auf die
Fixationspunkte ergibt sich die scharfe Gesamtwahrnehmung unserer Umgebung.

\subsection{Exkurs zur Selektiven Schärfe}

\begin{itemize}
\item Unscharfe Bildbereiche „vertreiben“ das Auge des Zuschauers
(keine Möglichkeit der scharfen Fixation ‐ selbst bei extremem Bemühen des Zuschauers)
\item Szenische Anwendung: „Selektive Schärfe“
Schärfe/Unschärfe wird klassisch ganz bewusst zur gezielten Lenkung und Bindung der Aufmerksamkeit der Zuschauer eingesetzt.
\end{itemize}

\textbf{Tiefenschärfe:}
Ist abhänging von: 
\begin{itemize}
\item Kamera‐/Focal‐Abstand a zur Szene
\item gewählte Brennweite \textit{f} 
\item Verhältnis Brennweite/Bildsensorgröße
\item Blendenzahl \textit{k} 
\end{itemize} 

\textbf{Funktionert das Prinzip der „selektiven Schärfe“ auch in VR-Produktionen?}
Durch die 3D-Umgebung hat der Mensch nun die Möglichkeit sich seinen Bereich von Interesse selbst zu suchen. Hier eine Unschärfe einzubauen würde eher zu einer negative empfundenen Erfahrung führen. \\

\subsection{Warum wir die Welt nicht doppelt sehen, oder Zyklopische Fusion}

\includegraphics[width=10cm]{Advanced Media Produktion/ZyklopischeFusion.JPG}

\begin{figure}[htbp] 
  \centering
     
  \caption{Zyklopische Fusion}
  \label{fig:Bild1}
\end{figure}


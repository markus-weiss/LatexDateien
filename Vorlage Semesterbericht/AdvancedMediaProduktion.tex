\chapter{Advanced Media Production}
\section{Einführung}\label{sec:Einführung_AMP}

S3D = Stereoskopische 3D \\

\section{S3D‐Aufnahme‐, Übertragungs‐ und Darstellungstechnologien}

\subsection{Aufnahmetechnologien}
\begin{itemize}

\item Stereo‐Rendering in 3D‐Programmen und Game‐Engines: Maya Renderer, Unity S3D-Renderer
\item S3D‐Realfilm‐Kamerakonzepte 2015:
\begin{itemize}
\item 21stCentury3D/Hyperstereo
\item GoPro Dual Hero
\item P+S/Freestyle‐Spiegelrig
\end{itemize}

\end{itemize}

\subsection{Betrachtungstechnologien}
\begin{description}
\item Polfilterverfahren: Linear \& Zirkular\\
\\
	Projektoren werfen ihr Licht durch entweder recht und links zirkulierende Polfilter bis hin zur reflektierenden, Silber beschichteten Leinwand. Das Bild wird nun zurück geworfen und trifft auf die Polfilterbrille wo nun das recht drehende Licht, des rechten Bildes durch den recht gerichteten Zirkularfilter läuft und somit auf das rechte Auge trifft. Auf der linken Seite ist der Vorgang identisch.
	 
\item RGB Wellenmultiplexverfahren (z.B. CAVE-Installation) \\
\\
Hier werden die Farbinformationen nach Wellenlängen getrennt um jeweils ein rechtes und ein linkes Halbbild zu erhalten.

\item 3DTV-Polfilterverfahren
Hier werden jeweils die geraden und ungeraden Zeilen in die entgegengesetzte Richtung zirkular ausgerichtet.

\item 3DTV-Shutterverfahren
Auf dem Display wird jeweils das rechte und im Anschluss das rechte Bild angezeigt. Mit einer über Infrarot synchronisierte Shutterbrille wird jeweils das eine oder andere Auge zeit synchron das Bild erhalten.

\item Anaglyphen-Verfahren
Die linke Seite der Brille ist cyan, die rechte magenta gefärbt ...?

%=============================%
\item autostereoskope Displays\\
\begin{description}
\item Prallax‐Barriere Display\\
...
\item Lentikular Display
...
\end{description}
Script Folie 107
\item VR-Headset
\end{description}

\subsection{Displayspeisung->Bildorganisation}
\begin{itemize}
\item Frame Packing \textit{progessiv, interlace} \\

\item Side-by-Side \textsubscript{(Half)*}

\item Top-and-Bottom \textsubscript{(squeezed)*}
\end{itemize}

\subsection{Unser Auge ist keine Kamera}
\begin{enumerate}
\item Pupille \& Iris\\
Diese öffnet sich bei schwachem, und schließt sich bei starkem Lichteinfall.

\item Muskuläre Steuerung\\
Unserer Augäpfel sind muskulär steuerbar (3 Freiheitsgrade,
s.a. Folie 28) und im Blickfeld frei beweglich, zusätzlich
unterstützt durch Kopfbewegungen (Folie 30).

\item Akkommodation\\
Die Linse in unserem Auge verändert ihre Brennweite durch muskuläre Dehnung/Streckung
Ziel: maximale Schärfeabbildung eines gewünschten Fixationspunkts auf die fovea centralis -> Akkommodation.
Fokussierungen -> 4m Abstand -> Ziliarkörper ist komplett entspannt. Linse ist in den Zonularfasern natürlich aufgespannt (dünn)
Fokussierungen < 2m Abstand führen zu deutlich „spürbaren“ Ziliarkontraktionen. Linse wird in den Zonularfasern entspannt (dick)

\item Rezeptive Felder\\
Die Rezeptoren auf der gekrümmten
Netzhaut sind zusammengeschaltet zu
„Rezeptiven Feldern“, welche mit zunehmenden
Abstand vom Sehzentrum
immer größer werden. Ein Nervenimpuls
wird nur bei Helligkeitsdifferenzen innerhalb
des Feldes ausgelöst.


\item Fovea zentralis\\
Zentraler Schärfebereich (fovea
centralis) und temporales / nasales
Gesichtsfeld werden im Gehirn
getrennt verarbeitet (s. Folie 34).

\end{enumerate}


\subsection{Natürliche Okulomotorik}

Unser natürliches Sehen ist geprägt von einem Konvergenz-zu-Akkomodations-Verhältnis (C/A)-Ratio von $\approx$ 1:1

\begin{itemize}

\item \textbf{Konvergenz:} Das Eindrehen der Augen
\item \textbf{Fixation:} Konvergieren zum Kreuzungspunkt
\item \textbf{Akkommodation:} Ist die Konvergenz im Gange versuchen \textit{gleichzeitig} Ziliarmuskeln den Brechungsindex der Augenlinsen Reflexhaft anzupassen
\item \textbf{Okulomotorische Stereopsis:} Zerebrales Feedback der muskulären Spannungen von Konvergenz \&	Akkommodation und Abgleich mit erlerntem, absolutem Entfernungswissen (Funktioniert aber nur bis ca. 2m Entfernung, Das Hirn kann sich Anhand der Muskelspannung den Abstand merken)

\end{itemize}







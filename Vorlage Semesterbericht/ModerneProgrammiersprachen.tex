\chapter{Moderne Programmiersprachen}
\section{Imperative Programmiersprachen}
\section{Hoare-Formeln}
\section{Hoare-Kalkül}

Vorbedingung, Programm, Nachbedingung\\

\subsection{Zuweisungsaxiom}
Zuweisung von Variablen\\

\subsection{Sequenzregel}
Ersetzbarkeit (transitivität) A{P}B - B{Q}C = A{PQ}C

\subsection{Alternativregel}
if
A und B {P} C
A und nicht B ->

Wenn A und B zutreffen wir P ausgeführt. Wenn A und nicht B zutreffen wird das Programm übersprungen.

\subsection{Iterationsregel}
Schleifeninvariaten I MUSS immer gleich bleiben, also vor und nach der Schleife 
die Schleife läuft so lange wie B = true solange bis B = false. Wenn B niemals false ist dann läuft die Schleife ewig.

\subsection{Konsequenzregel}
Man kann die Vorbedignung stärker und die nachbedigung schwächer und das programm gilt immer noch.

Stärker = stärker eingrenzen
Schwächer = weniger eingrenzen 

Wenn P das erste macht es auch das zweite:
x > 2 {P} x < 10
x > 3 {P} x < 10
x > 3 {P} x < 12






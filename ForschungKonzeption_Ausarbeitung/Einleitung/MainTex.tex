%Dokumentklasse
\documentclass[a4paper,12pt]{scrreprt}

% Dokumentinformationen
\usepackage[
	pdftitle={Ausarbeitung_Forschungskonzeption},
	pdfsubject={Ausarbeitung_Forschungskonzeption},
	pdfauthor={Markus Weiß},
	pdfkeywords={},	
	%Links nicht einrahmen
	hidelinks
]{hyperref}


% ============= Packages =============
\usepackage[utf8]{inputenc}
\usepackage[ngerman]{babel}
\usepackage{graphicx, subfig}
\usepackage{helvet}
\usepackage{textcomp}

% ============= PATHs =============
\graphicspath{{img/}}
\graphicspath{ {img/}{./img/} }

% ============= Kopf- und Fußzeile =============
\pagestyle{empty}

% ============= Dokumentbeginn =============
\begin{document}

%===========================TITEL-Seite==============================


\begin{titlepage}
    \begin{center}
    \huge \textbf{\textsf{Erstellung einer Shader-Modifikation unter Anwendung eines Generativ~Adversarial~Networks}} \\
    \vspace{2cm}
    	{\Large für das Computerspiel: „The Elder Scrolls V  \textsuperscript{\textcopyright}: Skyim“ \par}
    \vspace{2cm}	
    \LARGE\textbf{\textsc{Master-Thesis}}\\
    \vspace{1cm}
    \normalsize
    vorgelegt am: \today \\
    \vspace{2.5cm}
    \large \textbf{an der Fakultät Digitale Medien}\\
    \vspace{3cm}
    \end{center}
 \normalsize{
    \begin{tabular}{ll}
    	Name: & {Markus Weiß} \\
    	Matrikelnummer: & {249149} \\
		Fakultät: & Digitale Medien \\
    	Studiengang: & Medieninformatik Master\\
    	Studienjahrgang: & 2018\\
      Erstbetreuer: & {der erste} \\
      Zweitbetreuer: & {der zweite} \\
    \end{tabular}\\
    }
    \includegraphics[scale=1.5]{Logo_Hochschule_Furtwangen}
\end{titlepage}

%===========================Inhalt==============================

%Inhaltsverzeichnis
\tableofcontents

\include{Abstract}

\section{Zur Einleitung}
Im Folgenden ist der Weg zur Einleitung, wie auch die resultierende Einleitung aufgeführt.
\chapter{Checkliste}


\begin{itemize}
\item Hinführung
\item Ausgangssituation
\begin{itemize}
	\item Kurz das Thema schildern.
	\item Hierraus wird der Sollzustand abgeleitet werden können.
\end{itemize}
\item Problem 
\begin{itemize}
	\item Diskrepanz bzw. die Abweichung zwischen \textit{Soll} und \textit{Ist} 						verdeutlichen.
	\item Ist Situation formulieren. 
	\item Dient später zur Formulierung des Ziels.
\end{itemize}
\item Zielformulierung ( \textit{In einem Satz, wenn möglich keine Frage.} )
\begin{itemize}
	\item Abgleichen mit der Zielkategorie. Also: Beschreiben, Erkennen oder Gestalten.
	\item \textit{Soll}- Situation formulieren.
	\item Soll das oben genannte Problem lösen können. 
\end{itemize}
\item Vorhergehensweise
\begin{itemize}
	\item Beschreibung der Gliederung(Da die Gliederung ja den Roten Faden wiedergibt).
	\item Orientierung an der Gliederung.
	\item Offene Schrittfolgen Begründen.
\end{itemize}
\end{itemize}



\chapter{Einleitung}


\include{Gliederung}

\include{Methodenskizze}

\include{Forschungsbeitrag}

%Verzeichnis aller Bilder
%\listoffigures

%Verzeichnis aller Tabellen
%\listoftables

\end{document}
